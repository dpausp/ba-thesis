%% LyX 2.0.3 created this file.  For more info, see http://www.lyx.org/.
%% Do not edit unless you really know what you are doing.
\documentclass[11pt,a4paper,notitlepage]{scrbook}
\usepackage[T1]{fontenc}
\setlength{\parskip}{\medskipamount}
\setlength{\parindent}{0pt}
\usepackage[ngerman]{babel}

\makeatletter

\makeatother

\usepackage[a4paper,margin=2cm,tmargin=2cm]{geometry}

\usepackage{fontspec}
\setmainfont[Mapping = tex-text, FakeStretch = 1.04, WordSpace = 1.4]{FreeSans}

\usepackage{graphicx}
\usepackage{adjustbox}

\usepackage{xunicode}
\begin{document}

\pagenumbering{gobble}
\pagestyle{empty}

\subject{Bachelor-Arbeit im Fach Angewandte Informatik\\
zur Erlangung des akademischen Grades \\
Bachelor Of Science}


\title{i>PM 3D – Ein Prozessmodellierungswerkzeug für drei Dimensionen\\ %
\vspace{0.4cm}
Repräsentation von Prozessen im dreidimensionalen Raum – Konzept und Implementierung}


\author{vorgelegt von\\
\Large\textbf{Tobias Stenzel} \\
\vspace{1cm} \\
geboren am \\
28.02.1986 in \\
Weiden i.d.OPf.}


\date{%
\vspace{1cm}
angefertigt im Zeitraum von\\
10.10.2011 bis 10.4.2012}


\titlehead{%
 \begin{minipage}[t!]{8cm}
  Lehrstuhl für Angewandte Informatik IV\\
  Datenbanken und Informationssysteme\\
  Universität Bayreuth
 \end{minipage}
\hfill
 \begin{minipage}[t!]{4.8cm}
  \includegraphics[width=4.8cm]{Logo_gruen_eps}
 \end{minipage}
}

\maketitle

\vspace{2cm}
\line(1,0){400}

\textbf{Betreuer:}\smallskip{}
\\
Dr. Bernhard Volz\bigskip{}
\\
\textbf{Prüfer:}\smallskip{}
\\
Prof. Dr.-Ing. Stefan Jablonski\smallskip{}
\\
Prof. Dr. Bernhard Westfechtel
\\

\line(1,0){400}

\pagebreak

\section*{Zusammenfassung (deutsch)}

Prozessmodellierung ist eine etablierte Technik, welche von Unternehmen dazu genutzt wird, geschäftliche Abläufe zu planen, zu analysieren und weiter zu optimieren. 
Dafür eingesetzte Softwarewerkzeuge nutzen üblicherweise grafische 2D-Notationen zur Darstellung von Prozessdiagrammen.

3D-Prozessvisualisierungen wurden bislang nur relativ wenig untersucht. Insbesondere fehlt es an benutzbaren interaktiven Systemen zur 3D-Prozessmodellierung.

Das Projekt i>PM3D umfasst die prototypische Entwicklung eines flexiblen 3D-Modellierungswerkzeugs für die perspektivenorientierte Prozessmodellierung, welches auch neuartige (3D-)Eingabegeräte wie die Microsoft Kinect oder Nintendo Wii nutzt und die Anbindung von weiteren Eingabemöglichkeiten einfach macht.  

Gegenstand dieser Arbeit ist die Konzeption und Realisierung der Visualisierung von Prozessen in i>PM 3D sowie der internen Repräsentation der Modelle, welche in einer textuellen Form persistiert und wieder geladen werden kann. 
Die verwendete grafische Modellierungssprache wird über modifizierbare Metamodelle beschrieben und ist daher an spezielle Anforderungen anpassbar.
Im Rahmen dieser Arbeit entstand eine Render-Bibliothek auf Basis von OpenGL, welche auf die Erfordernisse der Prozessvisualisierung in i>PM 3D zugeschnitten ist, aber auch für andere Computergrafikanwendungen verwendet werden kann.


\pagebreak

\section*{Abstract (english)}

Process modeling is a widespread technique which is used for planning, analysis and optimization of business processes. 
Software tools for this purpose mostly use 2D notations for the display of process diagrams.

3D process visualizations haven't been studied extensively. 
Especially, usable interactive systems for business process modeling are missing.

The project called i>PM3D aims at developing a flexible 3D modeling tool for perspective oriented process modeling. 
Our tool also supports innovative 3D input devices like Microsoft Kinect or Nintendo Wii and makes it easy to add other input techniques. 

This thesis is concerned with the conception and realization of process visualization in i>PM3D. 
Additionally, it is shown how models are represented internally in our tool. 
Models can be saved to and loaded from a textual notation stored on disk. 
The language used for these models is described by modifiable meta models so it can be adapted to special use cases.
This work also includes a render library based on OpenGL, which was developed for the needs of i>PM3D but can also be used for other graphical applications.

\pagebreak

Ich versichere hiermit, die vorliegende Arbeit selbstständig verfasst und keine anderen als die
von mir angegebenen Quellen und Hilfsmittel benutzt sowie die Arbeit oder Teile davon nicht bereits zur Erlangung
eines akademischen Grades eingereicht zu haben.

Bayreuth, den 10. April 2012\\
\vspace{1.3cm} \\
\line(1,0){100}

(Tobias Stenzel)


\end{document}
